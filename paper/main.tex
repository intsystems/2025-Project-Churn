\documentclass{article}
\usepackage{arxiv}

\usepackage[utf8]{inputenc}
\usepackage[english, russian]{babel}
\usepackage[T1]{fontenc}
\usepackage{url}
\usepackage{booktabs}
\usepackage{amsfonts}
\usepackage{nicefrac}
\usepackage{microtype}
\usepackage{lipsum}
\usepackage{graphicx}
\usepackage{natbib}
\usepackage{doi}



\title{Нейросетевые подходы к решению задачи оттока абонентов}

\author{ Батарин Егор Владиславович \\
	Кафедра алгоритмов и технологий программирования \\
	Московский физико-технический институт\\
	Москва \\
	\texttt{batarin.ev@phystech.edu} \\
	%% examples of more authors
	\And
	Джумакаев Тимур Казбекович \\
	Мегафон\\
	Москва\\
}
\date{}

\renewcommand{\shorttitle}{\textit{arXiv} Template}

%%% Add PDF metadata to help others organize their library
%%% Once the PDF is generated, you can check the metadata with
%%% $ pdfinfo template.pdf
\hypersetup{
pdftitle={Нейросетевые подходы к решению задачи оттока абонентов},
pdfsubject={q-bio.NC, q-bio.QM},
pdfauthor={Батарин Егор Владиславович, Джумакаев Тимур Казбекович},
pdfkeywords={CatBoost, Модель Кокса, Анализ выживаемости},
}

\begin{document}
\maketitle

\begin{abstract}

В работе решается задача прогнозирования оттока абонентов компании Мегафон. Задача рассматривается как многоклассовая классификация, где в качестве меток класса выбраны факты оттока в будущие месяцы и факт отсутствия оттока в эти месяцы. Предлагаются различные подходы к решению задачи, как классические подходы: градиентный бустинг и модель Кокса, так и более современные подходы, связанные с применением методов глубокого обучения в моделях выживаемости. Проводится сравнение различных подходов с точки зрения принятых в работе критериев качества. В роли критериев качества модели выступают метрики Precision, Recall, F1, вычисленные при различных вероятностных порогах - числах, позволяющих перевести вероятности классов в метки классов. Цель работы заключается в поиске новых подходов к решению задачи оттока, которые покажут более высокие результаты по выбранным критериям качества, чем у текущих бейзлайнов. Эксперименты проведены на внутренних абонентских данных Мегафона.   

\end{abstract}


\keywords{CatBoost \and Модель Кокса \and Анализ выживаемости}

\section{Введение}

\section{Обзор литературы}







\bibliographystyle{unsrtnat}
\bibliography{references}

\end{document}