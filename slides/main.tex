\documentclass[10pt,pdf,hyperref={unicode}]{beamer}

\mode<presentation>
{
\usetheme{boxes}
\beamertemplatenavigationsymbolsempty

\setbeamertemplate{footline}[page number]
\setbeamersize{text margin left=0.5em, text margin right=0.5em}
}

\usepackage[utf8]{inputenc}
\usepackage[english, russian]{babel}
\usepackage{bm}
\usepackage{multirow}
\usepackage{ragged2e}
\usepackage{indentfirst}
\usepackage{multicol}
\usepackage{subfig}
\usepackage{amsmath,amssymb}
\usepackage{enumerate}
\usepackage{mathtools}
\usepackage{comment}
\usepackage{multicol}
\usepackage{graphicx}
\usepackage{subcaption}
\usepackage[all]{xy}

\usepackage{tikz}
\usetikzlibrary{positioning,arrows}

\tikzstyle{name} = [parameters]
\definecolor{name}{rgb}{0.5,0.5,0.5}

\usepackage{caption}
\captionsetup{skip=0pt,belowskip=0pt}

\newtheorem{rustheorem}{Теорема}
\newtheorem{russtatement}{Утверждение}
\newtheorem{rusdefinition}{Определение}

% colors
\definecolor{darkgreen}{rgb}{0.0, 0.2, 0.13}
\definecolor{darkcyan}{rgb}{0.0, 0.55, 0.55}

\AtBeginEnvironment{figure}{\setcounter{subfigure}{0}}

\captionsetup[subfloat]{labelformat=empty}

%----------------------------------------------------------------------------------------------------------

\title[Анализ выживаемости]{Анализ выживаемости}
\author{Е.\,В.\,Батарин}

\institute[]{Московский физико-технический институт}
\date[2022]{\small 27\;марта\;2025\,г.}

%---------------------------------------------------------------------------------------------------------
\begin{document}

\begin{frame}
\titlepage
\end{frame}

%----------------------------------------------------------------------------------------------------------
\section{Слайд об исследованиях}
\begin{frame}{Слайд об исследованиях}
\bigskip
Исследуется проблема анализа выживаемости абонентов
\begin{block}{Цель исследования~---}
предложить метод \ldots.
\end{block}
\begin{block}{Требуется предложить}
\justifying
\begin{enumerate}[1)]
\item метод \ldots,
\item метод \ldots,
\item метод \ldots.
\end{enumerate}
\end{block}
\begin{block}{Решение}
Для \ldots.
\end{block}
\end{frame}

%---------------------------------------------------------------------------------------------------------
\section{Постановка задачи анализа выживаемости}
\begin{frame}{Постановка задачи анализа выживаемости}

$\mathcal{T}=\{0,\ldots,T_{\max}\}$ - дискретное время

$\mathcal{K}=\{\emptyset,1,\cdots,K\}$ - множество событий

$\tau^i=\min(T^i,C^i) \in\mathcal{T} $ - право-цензурированные отсчеты времени

$\mathcal{X}^i(t)=\{\mathbf{x}^i(t_j^i):0\leq t_j^i\leq t\mathrm{~for~}j=1,\cdots,J^i\}$ - вектора признаков 

$\mathcal{D}=\{(\mathcal{X}^i,\tau^{i},k^{i})\}_{i=1}^{N}$ - обучающая выборка



{\fontsize{8.5}{10}\selectfont
	\begin{align*}
		& \begin{aligned}
			F_{k^{*}}(\tau^{*}|\mathcal{X}^{*}) & = P(T\leq\tau^{*},k=k^{*}|\mathcal{X}^{*},T>t_{J^{*}}^{*}) \\
			& =\sum_{\tau\leq\tau^*}P(T=\tau,k=k^*|\mathcal{X}^*,T>t_{J^*}^*).
		\end{aligned}
		\quad
		& \begin{aligned}
			S(\tau^{*}|\mathcal{X}^{*}) & = P(T>\tau^*|\mathcal{X}^{*},T>t_{J^*}^{*}) \\
			& =1-\sum_{k\neq\emptyset}F_k(\tau^*|\mathcal{X}^{*}).
		\end{aligned} \\
		% Подписи под формулами без смещения формул
		& \text{Функция распределения для события $k^*$}
		& \hspace{55pt} \text{Функция выживания}
	\end{align*}
}



\begin{figure}[h!]
	\centering
	\begin{minipage}{0.45\textwidth}
		\centering
		\includegraphics[width=\linewidth]{../figures/survival_example.png}
		\caption{Пример цензурирования}
	\end{minipage}
	\hfill
	\begin{minipage}{0.45\textwidth}
		\centering
		\includegraphics[width=\linewidth]{../figures/dynamic_deephit_scheme_only_architecture.png}
		\caption{Архитектура модели}
	\end{minipage}
\end{figure}

\end{frame}

%----------------------------------------------------------------------------------------------------------
\section{Предложенный метод \ldots}
\begin{frame}{Предложенный метод \ldots}
~\\[-1mm]
Заданы
\begin{enumerate}[1)]
	\item \ldots,
	\item \ldots.
\end{enumerate}

\medskip
Параметрические семейства:
\[
\setlength\abovedisplayskip{0pt}
\mathfrak{F} = \left\{\mathbf{f}| \mathbf{f} = \text{softmax}\bigr(\mathbf{v}\bigr(\mathbf{x}\bigr)/T\bigr), \quad \mathbf{v}: \mathbb{R}^{n} \to \mathbb{R}^K \right\},
\]
\[
\mathfrak{G} = \left\{\mathbf{g}| \mathbf{g} = \text{softmax}\bigr(\mathbf{z}\bigr(\mathbf{x}\bigr)/T\bigr), \quad \mathbf{z}: \mathbb{R}^n \to \mathbb{R}^K \right\},
\]
где~\ldots.

\medskip
Функция ошибки
\[
\setlength\abovedisplayskip{0pt}
\begin{aligned}
   \mathcal{L}\bigr(\mathbf{g}\bigr) = &-\sum_{i=1}^{m}\underbrace{{\sum_{k=1}^{K}y^k_i\log\mathbf{g}\bigr(\mathbf{x}_i\bigr)\bigr|_{T=1}}}_{\text{исходная функция потерь}}- \sum_{i=1}^{m}\underbrace{{\sum_{k=1}^{K}\mathbf{f}\bigr(\mathbf{x}_i\bigr)\bigr|_{T=T_0}\log\mathbf{g}\bigr(\mathbf{x}_i\bigr)\bigr|_{T=T_0}}}_{\text{слагаемое дистилляции}},
\end{aligned}
\setlength\belowdisplayskip{0pt}
\]
где~\ldots.

Оптимальная модель выбирается из класса,
$\hat{\mathbf{g}} = \arg\min\limits_{\mathbf{g} \in \mathfrak{G}_{\text{cl}}} \mathcal{L}\bigr(\mathbf{g}\bigr).$

\bigskip
\footnotetext[1]{\textit{Lopez-Paz D., Bottou L., Scholkopf B., Vapnik V.} Unifying distillation and privileged information // ICLR, 2016.}
\footnotetext[2]{\textit{Hinton G., Vinyals O., Dean J.} Distilling the knowledge in a neural network // NIPS, 2015.}
\end{frame}

%----------------------------------------------------------------------------------------------------------
\section{Анализ предложенного метода \ldots}
\begin{frame}{Анализ предложенного метода \ldots}
\justifying

На графике показана зависимость значения параметров$w_i$ в зависимости от параметра~$l_1$-регуляризации~$C$.


С увеличением параметра регуляризации~$C$ число ненулевых параметров~$w_i$ уменьшается.

\end{frame}

%----------------------------------------------------------------------------------------------------------
\section{Выводы}
\begin{frame}{Выводы}
\justifying
	\begin{enumerate}
	\justifying
	    \item Предложен \ldots.
        \item Доказаны теоремы \ldots, 
        \begin{itemize}
            \item[---] \ldots,
            \item[---] \ldots.
        \end{itemize}
        \item Предложен метод \ldots
        \begin{itemize}
            \item[---] \ldots,
            \item[---] \ldots.
        \end{itemize}
        \item Предложены методы \ldots
        \begin{itemize}
            \item[---] \ldots,
            \item[---] \ldots.
        \end{itemize}
        \item Предложена вероятностная интерпретации \ldots.
	\end{enumerate}
\end{frame}
%----------------------------------------------------------------------------------------------------------

\end{document} 